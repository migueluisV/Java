\section{Conceptos básicos}
\begin{itemize}
    \item Todos los programas de Java están contenidos dentro de una \textbf{clase} con algún nombre en particular, como \textbf{App}, \textbf{Program} o un nombre personalizado.
    \item Todos los programas de Java tienen un punto de partida, un procedimiento llamado \textbf{main} (como en C\#).
    \item El lenguaje incluye temas como:
    \begin{itemize}
        \item Indicadores de acceso.
        \item Tipo de retorno de métodos (procedimientos y funciones).
        \item Formas de creación de métodos (estática o no).
        \item Parámetros en funciones y procedimientos.
    \end{itemize}
    \item Cada instrucción debe terminar con un punto y coma (\textbf{;}).
    \item Cada clase, función u objeto debe abrirse con llaves (\textbf{\{ \}}.
\end{itemize}
Esta es la estructura inicial o básica de un programa en Java:
\begin{lstlisting}
    public class App {
        public static void main(String[] args) throws Exception {
            System.out.println("hola mundo");
        }
}
\end{lstlisting}



\section{Comentarios}
Java soporta el poder hacer comentarios de una y varias líneas:\\
Se usan dos diagonales (\textbf{//}) para comentar una línea de código.\begin{center}\textit{//Esto es un comentario\\x = 5; //Variable comentada}\end{center}
Se usa una diagonal y un asterisco (\textbf{/*}) al inicio del bloque y un asterisco y diagonal (\textbf{*/}) al final para comentar varias líneas o un bloque de código.
\begin{center}\textit{/*System.out.println("Hola mundo");\\x = 5;\\Dos líneas comentadas*/}\end{center}
Una herramienta del \textit{Java Development Kit} es que se puede generar documentación automática de un proyecto Java en formato HTML, esta documentación toma el código fuente y lo muestra más presentable para que otro desarrollador entienda el programa. Utilizando los \textbf{comentarios de documentación} podemos hacer que esta documentación generada esté aún más presentable, por así decirlo, si insertamos comentarios de documentación en cada clase, función, procedimiento, declaración de variable, cuando generemos la documentación, veremos qué ocurre en cada parte donde pusimos un comentario de este tipo. Para insertar un comentario de documentación, utilizamos una diagonal y dos asteriscos (\textbf{/**}) al inicio de la instrucción y una diagonal y un asterisco al final (\textbf{*/}). \begin{center}\textit{/** Inicia una clase */\\class Prueba \{\}}\end{center}
Por último, si escribimos una diagonal y más de un asterisco (\textbf{/*******...}), Java lo interpreta como que se quiere hacer un recuadro o una separación entre partes o bloques de código, como se puede ver a continuación:\begin{center}\textit{/**************************\\Comienzo de un método\\/**************************}\end{center}



\section{Clases importadas}
Podemos importar clases especiales u externas para volver más rico el programa de java, esto es como las librerías o cabeceras en otros lenguajes.\\
La palabra reservada a utilizar es \textbf{import}, seguido de la clase a importar, por ejemplo: java.util.Scanner.



\section{Variables}
Los tipos de datos existentes en Java son:
\begin{itemize}
    \item \textbf{int}: datos numéricos enteros.
    \item \textbf{double}: datos numéricos con punto decimal.
    \item \textbf{String}: una cadena de texto.
    \item \textbf{char}: un único carácter.
    \item \textbf{boolean}: boleano, almacena si o no.
\end{itemize}
Recordemos que hay variaciones con estas variables, podemos crear una variable entera que tenga menos espacio de bits para almacenar un número (byte y short), lo mismo ocurre con los flotantes (float) o hacer que acepten muchos bits (long).\\
Se puede poner una coma entre cada variable declarada del mismo tipo para hacer una declaración de variables en una sola línea:\begin{center}\textit{double n1, n2, n3, n4, n5, prom;}\end{center}

\subsection{Concatenar cadenas}
Se utiliza el \textbf{operador +} para concatenar (unir) más de una cadena.\begin{center}\textit{System.out.println("hola" + "mundo" + "en" + "Java");}\end{center}

\subsection{Reglas para nombrar variables}
Para administrar mejor el nombre de las variables se pueden seguir los siguientes consejos:
\begin{itemize}
    \item Ponerle una letra mayúscula o un guión bajo (\_).
    \item Ponerle un nombre significativo que se pueda recordar fácilmente.
    \item No se permiten caracteres especiales o espacios en blanco.
    \item No se permiten palabras reservados como nombres de variables.
    \item C++ distingue entre mayúsculas y minúsculas, para tener cuidado nombrando así las variables.
\end{itemize}



\section{Operadores primitivos}
Las operaciones matemáticas que se pueden realizar en Java son las básicas: \textbf{igualdad} (=), \textbf{suma} (+), \textbf{resta} (-), \textbf{multiplicación} (*), \textbf{división} (/) y \textbf{modulo} (\%), estas operaciones se realizan de la misma manera en la que se hacen en ecuaciones u operaciones algebraicas.
\begin{center}
    \textit{int num1 = 1, num2 = 2, num3 = 3;\\int suma = num1 + num2 + num3;\\int resta = num3 - num1 - num2;\\int multi = 5 * 4 * num3; //Ejemplos de operaciones matemáticas con los operandos\\int div = num1 / num2; //El valor almacenado en div es un entero, si se busca que se almacene el valor decimal real se requiere un float o double.\\int mod = num2 \% 2; //Almacena el residuo de una división}
\end{center}

\subsection{Operador de incremento y decremento}
Como el otros lenguajes, puedes utilizar los operadores de incremento y decremento en variables, se pueden realizar estas operaciones por medio de un \textbf{prefijo} o \textbf{posfijo}:
\begin{itemize}
    \item Con prefijo, primero se incrementa o decrementa el valor de la variable, luego se utiliza en un ciclo u operación que se de.
    \begin{center}\textit{int x = 5;\\int y = ++x; //Incrementa x a 6 y a la variable y se le asigna 6}\end{center}
    \item Con posfijo, primero se utiliza el valor de la variable en un ciclo u operación, luego se incrementa o decrementa el valor.
    \begin{center}\textit{int x = 5;\\int y = x++; //A la variable y se le asigna 5 y después incrementa x a 6}\end{center}
\end{itemize}

\subsection{Operador de asignación}
Este operador se utiliza en operaciones o variables acumulativas (como para sumar varias calificaciones y luego calcular un promedio) donde a una variable se le asigna su propio contenido, más otro valor, o para incrementar o decrementar una variable más de una unidad o 1, estos operadores de asignación aplican para los operandos existentes:
\begin{center}
    \textit{int os = 1, or = 2, om = 3, od = 4, om = 5;\\os += 1; //os = 1 + 1\\or -= 3; //or = 2 - 3\\om *= 5; //om = 3 * 5\\od /= 7; //od = 4 / 7\\om \%= 9; //om = 5 \% 9}
\end{center}



\section{Asignación de valores a variables}
Para dar una entrada de datos en Java, existen varias formas, en esta ocasión, utilizaremos una clase importada para realizar dicha acción. Se debe agregar una clase especial para realizar entrada de datos en un proyecto de Java: la clase \textbf{Scanner} y también debemos crear un objeto de la misma para trabajar con esta clase:\begin{center}\textit{import java.util.Scanner;\\Scanner var = new Scanner(System.in);}\end{center}
Esta clase contiene distintos procedimientos para leer entradas de un tipo de dato específico:
\begin{itemize}
    \item \textbf{nextByte()}: lee un entero byte.
    \item \textbf{nextShort()}: lee un entero short.
    \item \textbf{nextInt()}: lee un entero.
    \item \textbf{nextLong()}: lee un decimal long.
    \item \textbf{nextFloat()}: lee un decimal float.
    \item \textbf{nextDouble()}: lee un decimal double.
    \item \textbf{nextBoolean()}: lee una entrada boleana.
    \item \textbf{nextLine()}: lee toda una cadena de texto.
    \item \textbf{next()}: lee únicamente una palabra de una cadena con varias palabras.
\end{itemize}
\begin{lstlisting}
    import java.util.Scanner; //Importa la clase al proyecto

    class App {
        public static void main(String[ ] args) {
            String variable;
            Scanner myVar = new Scanner(System.in); //Crea un objet de la clase importada
            variable = myVar.nextLine(); //Asigna lo que se lea en la entrada a una variable tipo string
            System.out.println(myVar.nextLine()); //Despliega lo que se lea en la entrada de datos.
            System.out.println(variable); //Despliega la variable string
        }
    }
\end{lstlisting}

\subsection{Despliegue de información y formato}
Podemos ver en el código anterior que está la siguiente instrucción: \textit{System.out.println(variable);}, esta instrucción es utilizada para imprimr cadenas o variables en pantalla, \textbf{System.out} es la clase base que se utiliza, sin embargo, \textbf{println} escribe una cadena o variable en pantalla y da un salto de línea automáticamente, si buscamos que no se dé ningún salto de línea, podemos utilizar el método \textbf{print()}, que escribe una cadena o variable sin dar un salto de línea, bastante sencillo, \textbf{printf(formato, argumentos)} te permite dar formatos a cadenas, como se ve en el siguiente código en el lenguaje C\#:
\\begin{center}\textit{Console.WriteLine("\{0\} - \{1\}", var1, var2);}\end{center}
Si podemos apreciar, la cadena desplegada con \textit{Console.WriteLine} tiene un formato, donde los números son sustituidos por variables, podemos aplicar esto mismo utilizando \textit{printf(formato, argumentos)}, como se ve en el siguiente código:
\begin{center}\textit{System.out.printf("\%s - \%s", var1, var2);}\end{center}
Dejo las distintas opciones de formatos en JAVA:
\begin{itemize}
	\item \textbf{\%b}: escribe un booleano.
	\item \textbf{\%h}: escribe un Hascode.
	\item \textbf{\%s}: escribe una cadena.
	\item \textbf{\%c}: escribe una cadena unicode.
	\item \textbf{\%d}: escribe un entero decimal.
	\item \textbf{\%o}: escribe un entero octal.
	\item \textbf{\%x}: escribe un entero hexadecimal.
	\item \textbf{\%f}: escribe un real decimal.
	\item \textbf{\%e}: escribe un real con notación científica.
	\item \textbf{\%g}: escribe un real con notación científica o decimal.
	\item \textbf{\%a}: escribe un real hexadecimal con mantisa y exponente.
	\item \textbf{\%t}: escribe una fecha u hora.
\end{itemize}

\subsection{Limpieza de Buffer de entrada}
En ocasiones, cuando ingresamos múltiples datos de distintos tipos en variables utilizando el mismo objeto \textit{Scanner}, en este ocurre un conflicto de entrada, donde el Buffer mantiene el último dato ingresado en vez de abrir la posibilidad de ingresar otro; la situación regular que desearíamos sería la siguiente:
\begin{lstlisting}
    import java.util.Scanner; //Importa la clase al proyecto

    class App {
        public static void main(String[ ] args) {
        		String nombre;
        		int edad;
        		Scanner lectura = new Scanner(System.in);
        		
        		System.out.print("Ingrese su nombre: "); nombre = lectura.nextLine();
        		System.out.print("Ingrese su edad: "); edad = lectura.nextInt();
        		
        		System.out.println("Su nombre es: " + nombre);
        		System.out.println("Su edad es: " + edad);
        		
        		//Despliega:
        		//Su nombre es: luis
        		//Su edad es: 21
        }
    }
\end{lstlisting}
Sin embargo, si cambiamos el orden de la asignación de valores a las variables del código anterior (\textit{nextLine()} después de \textit{nextInt()}), es decir, primero ingresamos un valor número y después una cadena, ocurre la siguiente situación:
\begin{lstlisting}
    import java.util.Scanner; //Importa la clase al proyecto

    class App {
        public static void main(String[ ] args) {
        		String nombre;
        		int edad;
        		Scanner lectura = new Scanner(System.in);
        		
        		System.out.print("Ingrese su edad: "); edad = lectura.nextInt();
        		System.out.print("Ingrese su nombre: "); nombre = lectura.nextLine();
        		
        		System.out.println("Su nombre es: " + nombre);
        		System.out.println("Su edad es: " + edad);
        		
        		//Despliega:
        		//Su nombre es: Su edad es: 21
        }
    }
\end{lstlisting}
Entonces, vemos que el programa se salta la parte del ingreso de un valor a la variable de cadena, esto es debido a que el objeto Scanner tiene un conflicto en su buffer, para solucionarlo, escribimos la instrucción \textbf{lectura.nextLine()} entre ambas asignaciónes de variables:\begin{center}\textit{System.out.print("Ingrese su edad: "); edad = lectura.nextInt();\\lectura.nextLine();\\System.out.print("Ingrese su nombre: "); nombre = lectura.nextLine();}\end{center}
Esta instrucción recibe lo que sea que esté en el buffer del objeto Scanner, dejando el paso libre a la siguiente asignación de valor y evita este conflicto de buffer.

\subsection{Pausar un programa}
Podemos pausar el flujo de un programa utilizando el comando \textbf{System.in.read()} (proveniente de la clase \textbf{java.io.IOException}) posicionandolo en la sección de código donde deseemos pausarlo, como vemos en el siguiente código:
\begin{lstlisting}
	import java.io.IOException;

    class App {
        public static void main(String[ ] args) {
        		System.out.print("Mensaje 1");
        		System.in.read(); //Espera a que se presione ENTER para finalizar
        		System.out.print("Mensaje 2");
        		System.in.read(); //Espera a que se presione ENTER para finalizar
        		System.out.print("Mensaje 3");
        		System.in.read(); //Espera a que se presione ENTER para finalizar
        }
    }
\end{lstlisting}
\textbf{Advertencia}: después de experimentar con este método durante algunos programas, he visto que causa conflicto con la clase Scanner, lanzando excepciones imprevistas, use esta herramienta bajo su propio riesgo.



\section{Terminar la ejecución de un programa}
Si deseamos finalizar la ejecución de un programa, podemos utilizar la instrucción \textbf{Systen.exit()}, dentro de los paréntesis, va un valor entero el cual indicará cómo es que terminará la ejecución, donde \textbf{0 indica una finalización éxitosa} y sin errores, \textbf{1} o \textbf{-1 para indicar una finalización con errores} o mensaje de error.
\begin{lstlisting}
    class App {
        public static void main(String[ ] args) {
        		System.out.print("Presione ENTER para continuar...");
        		System.in.read(); //Espera a que se presione ENTER para finalizar
        		System.exit(0); //Finaliza el programa
        		
        		//Despliega:
        		//Presione ENTER para continuar...
        		//Process finished with exit code 0
        }
    }
\end{lstlisting}



\section{Limpieza de pantalla}
Si deseamos que, al cumplir alguna condición o después de presionar alguna tecla, todo el contenido de nuestra consola, terminal o pantalla se borre, en JAVA existen algunas opciones para lograr esta tarea, después de probarlas, la que nos funcionó es la siguiente: debemos utilizar el \textbf{intérprete de línea de comandos} (CMD por sus siglas en inglés) para utilizar su comando \textbf{cls}, no explicaremos a detalle como es que funciona la siguiente instrucción, usted puede escribirla donde desee que se ejecute, y verá la tarea resuelta:
\begin{lstlisting}
	import java.io.IOException; //Clase necesaria para utilizar CMD en JAVA

    class App {
    		//En la clase donde se vaya a utilizar el CMD, debe lanzar (throw) las excepciones IOException y InterruptedException
        public static void main(String[ ] args) throws IOException, InterruptedException{
        		System.out.print("Hola mundo 1"); //Despliega un mensaje
        		new ProcessBuilder("cmd", "/c", "cls").inheritIO().start().waitFor(); //Instrucción para limpiar pantalla
        		System.out.print("Hola mundo 2"); //Despliega otro mensaje
        }
    }
\end{lstlisting}